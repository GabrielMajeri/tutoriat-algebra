\documentclass[11pt]{article}

\usepackage{fontspec}

\usepackage{polyglossia}
\setdefaultlanguage{romanian}

\usepackage{amsmath}
\usepackage{amsthm}
\usepackage{amssymb}
\usepackage{unicode-math}

\usepackage{braket}

\theoremstyle{definition}
\newtheorem*{exercise}{Exercițiu}

\begin{document}

\begin{exercise}
Găsiți elementele inversabile, divizorii lui 0, elementele nilpotente și elementele idempotente din \(\mathbb{Z}_{63}\).
\end{exercise}
\begin{proof}
~
\begin{itemize}
    \item Elementele au invers la înmulțire în \(\mathbb{Z}_n\) dacă și numai dacă sunt prime față de \(n\). În acest caz, \(U(\mathbb{Z}_{63}) = \Set{ \hat{1}, \hat{2}, \hat{4}, \hat{5}, \dots }\).
    
    \item Divizorii lui 0 dintr-un inel \(R\) sunt elementele \(a \in R\) pentru care \(\exists b \neq 0\) astfel încât \(ab = 0\). Din acest motiv, elementele care sunt inversabile sigur nu pot fi divizori ai lui 0.

    În \(\mathbb{Z}_n\), toate numerele care au un factor în comun cu \(n\) sunt divizori ai lui 0. Răspunsul este \(\Set{ \hat{3}, \hat{6}, \hat{7}, \hat{9}, \hat{12}, \dots }\).
    
    \item Elementele nilpotente sunt cele pentru care \(\exists n \in \mathbb{N}^*\) astfel încât \(a^n = 0\). 0 este întotdeauna nilpotent, dar este posibil să mai existe și alte elemente de acest fel.
    
    În \(\mathbb{Z}_n\), elementele nilpotente sunt cele care conțin cel puțin toți factorii primi distincți ai lui \(n\). Pentru \(n = 63\), avem \(63 = 3^2 \cdot 7\), deci trebuie să găsim numere care să fie multiplii de \(3 \cdot 7\). Răspunsul este \(\mathcal{N}(\mathbb{Z}_{63}) = \Set{ \hat{0} (= 3 \cdot 7 \cdot 0), \hat{21} (= 3 \cdot 7 \cdot 1), \hat{42} (= 3 \cdot 7 \cdot 2) }\).
    
    \item Elementele idempotente sunt cele pentru care \(a^2 = a\). Atât 0 cât și 1 sunt întotdeauna idempotente. De asemenea, dacă \(a\) este idempotent, atunci și \(1 - a\) este idempotent, deci odată ce am găsit jumătate dintre idempotente putem obține mai ușor și cealaltă jumătate.
    
    Ideea este să descompunem \(n\) în produs de \(r\) numere prime sau puteri de numere prime, \(n = p_1^{k_1} p_2^{k_2} \dots p_r^{k_r}\) și să descompunem \(\mathbb{Z}_n\) în \(\mathbb{Z}_{p_1^{k_1}} \times \mathbb{Z}_{p_2^{k_2}} \times \dots \times \mathbb{Z}_{p_r^{k_r}}\).
    Singurele elemente idempotente din fiecare \(\mathbb{Z}_{p^k}\) sunt \(\bar{0}\) și \(\bar{1}\).
    
    În acest caz \(\mathbb{Z}_{63} = \mathbb{Z}_{7} \times \mathbb{Z}_{9}\).
    Trebuie să scriem toate cele \(2^r\) șiruri posibile de 0 și 1 de lungime \(r\) (fiecare corespunde unui idempotent):
    \begin{itemize}
        \item \((\bar{0}, \bar{\bar{0}})\) \(\implies\) numere care dau rest 0 la împărțirea cu 7 și rest 0 la împărțirea cu 9 \(\implies\) \(\hat{0}\)
        \item \((\bar{1}, \bar{\bar{1}})\) \(\implies\) numere care dau rest 1 la împărțirea cu 7 și rest 1 la împărțirea cu 9 \(\implies\) \(\hat{1}\)
        \item \((\bar{0}, \bar{\bar{1}})\) \(\implies\) numere care dau rest 0 la împărțirea cu 7 și rest 1 la împărțirea cu 9 \(\implies\) \(\hat{28}\)
        \item Din moment ce deja știm că \(\hat{28}\) este idempotent, știm că și \(1 - 28 \equiv -27 \equiv 36 \mod 63\) este idempotent. Dacă nu observăm acest lucru, putem să calculăm ca mai înainte:
        
        \((\bar{1}, \bar{\bar{0}})\) \(\implies\) numere care dau rest 1 la împărțirea cu 7 și rest 0 la împărțirea cu 9 \(\implies\) \(\hat{36}\)
    \end{itemize}
    
    Deci \(\text{Idemp}(\mathbb{Z}_{63}) = \Set{ \hat{0}, \hat{1}, \hat{28}, \hat{36} }\).
\end{itemize}

\end{proof}

\begin{exercise}
Găsiți idealele lui \(\mathbb{C} \times M_4(\mathbb{Z}_6 \times \mathbb{R})\).
\end{exercise}
\begin{proof}
Ne folosim de una sau mai multe dintre următoare proprietăți (în rezolvare, trebuie enunțate înainte de folosirea lor):
\begin{itemize}
    \item Idealele lui \(R_1 \times R_2 \times \dots \times R_n\) sunt \(I_1 \times I_2 \times \dots \times I_n\), unde \(I_1, I_2, \dots, I_n\) sunt ideale ale lui \(R_1, R_2, \dots, \text{ respectiv } R_n\).
    \item Idealele lui \(\mathbb{Z}\) sunt \(n\mathbb{Z}, \forall n \in \mathbb{N}^{*}\).
    \item Idealele lui \(\mathbb{Z}_n\) sunt \(\hat{d}\mathbb{Z}_n\) unde \(d \mid n\).
    \item Dacă \(R\) este corp, atunci singurele lui ideale sunt \(\Set{0}\) și \(R\).
    \item Idealele lui \(M_n(R)\) sunt \(M_n(I)\) unde \(I\) este un ideal al lui \(R\).
\end{itemize}

În acest caz:
\begin{itemize}
    \item Fiind corpuri, idealele lui \(\mathbb{C}\) și \(\mathbb{R}\) sunt \(\Set{0}\) și ele însăși.
    \item Idealele lui \(\mathbb{Z}_6\) sunt \(\hat{d}\mathbb{Z}_6\) unde \(d \mid 6\), deci \(\hat{1}\mathbb{Z}_6, \hat{2}\mathbb{Z}_6, \hat{3}\mathbb{Z}_6, \hat{6}\mathbb{Z}_6\).
    \item Ne folosim de prima proprietate pentru idealele produsului de inele.
    \item Ne folosim de ultima proprietate pentru idealele inelelor de matrici.
\end{itemize}

În concluzie, idealele căutate sunt:
\begin{itemize}
    \item \(\Set{0} \times M_4(\hat{1} \mathbb{Z}_6 \times \set{0})\)
    \item \(\Set{0} \times M_4(\hat{1} \mathbb{Z}_6 \times \mathbb{R})\)
    \item \(\Set{0} \times M_4(\hat{2} \mathbb{Z}_6 \times \set{0})\)
    \item \(\Set{0} \times M_4(\hat{2} \mathbb{Z}_6 \times \mathbb{R})\)
    \item \dots
    \item \(\mathbb{C} \times M_4(\hat{6} \mathbb{Z}_6 \times \set{0})\)
    \item \(\mathbb{C} \times M_4(\hat{6} \mathbb{Z}_6 \times \mathbb{R})\)
\end{itemize}
\end{proof}

\begin{exercise}
Găsiți idealele inelului \(R = \mathbb{Z}_8 \times \mathbb{Q}\) și, până la izomorfism, inelele factor obținute prin împărțirea la aceste ideale.
\end{exercise}
\begin{proof}
Asemănător exercițiului anterior, găsim idealele acestui inel: \(\hat{1} \mathbb{Z}_8 \times \Set{0}, \hat{1} \mathbb{Z}_8 \times \mathbb{Q}, \dots, \hat{8}\mathbb{Z}_8 \times(\Set{0}), \hat{8}\mathbb{Z}_8 \times \mathbb{Q}\).

Pentru a doua parte ne vom folosi și de următoarele proprietăți:
\begin{itemize}
    \item Dacă \(I_1 \trianglelefteq R_1\) și \(I_2 \trianglelefteq R_2\), \((R_1 \times R_2) / (I_1 \times I_2) \cong (R_1 / I_1) \times (R_2 / I_2)\)
    \item Dacă \(I, J \trianglelefteq R\) și \(J \subset I\), atunci \((R / J) / (I / J) \cong (R / I)\).
    \item \((R / R) \cong \Set{0}, (R / \Set{0}) \cong R\)
    \item \(\Set{0} \times R \cong R \times \Set{0} \cong R\)
    \item \(\mathbb{Z}_n \cong \mathbb{Z} / n\mathbb{Z}\)
\end{itemize}

Acum trebuie să scriem toate inelele factor de forma \(\frac{\mathbb{Z}_8 \times \mathbb{Q}}{I \times J}\), unde \(I \in \Set{ \hat{1}\mathbb{Z}_8, \dots, \hat{8}\mathbb{Z}_8 }\) și \(J \in \Set{ \Set{0}, \mathbb{Q} }\).
Trebuie de asemenea să ``simplificăm'' aceste inele până la forma în care se vede care sunt izomorfe între ele.
\begin{itemize}
    \item \[
    \frac{\mathbb{Z}_8 \times \mathbb{Q}}{\hat{1} \mathbb{Z}_8 \times \Set{0}}
    \cong \frac{\mathbb{Z}_8}{\mathbb{Z}_8} \times \frac{\mathbb{Q}}{\Set{0}}
    \cong \Set{\hat{0}} \times \mathbb{Q} \cong \mathbb{Q}
    \]
    \item \[
    \frac{\mathbb{Z}_8 \times \mathbb{Q}}{\hat{1} \mathbb{Z}_8 \times \mathbb{Q}}
    \cong \frac{\mathbb{Z}_8}{\mathbb{Z}_8} \times \frac{\mathbb{Q}}{\mathbb{Q}}
    \cong \Set{\hat{0}} \times \Set{0} \cong \Set{0}
    \]
    \item \[
    \dots
    \]
    \item \[
    \frac{\mathbb{Z}_8 \times \mathbb{Q}}{\hat{4} \mathbb{Z}_8 \times \mathbb{Q}}
    \cong \frac{\mathbb{Z}_8}{\hat{4}\mathbb{Z}_8} \times \frac{\mathbb{Q}}{\mathbb{Q}}
    \cong \frac{\mathbb{Z} / 8\mathbb{Z}}{4\mathbb{Z} / 8\mathbb{Z}} \times \Set{0}
    \cong \frac{\mathbb{Z}}{4\mathbb{Z}} \times \Set{0}
    \cong \mathbb{Z}_4 \times \Set{0}
    \cong \mathbb{Z}_4
    \]
    \item \[
    \dots
    \]
    \item \[
    \frac{\mathbb{Z}_8 \times \mathbb{Q}}{\hat{8} \mathbb{Z}_8 \times \mathbb{Q}}
    \cong \frac{\mathbb{Z}_8}{\hat{8}\mathbb{Z}_8} \times \frac{\mathbb{Q}}{\mathbb{Q}}
    \cong \frac{\mathbb{Z}_8}{\Set{0}} \times \Set{0}
    \cong \mathbb{Z}_8 \times \Set{0}
    \cong \mathbb{Z}_8
    \]
\end{itemize}
\end{proof}

\end{document}
