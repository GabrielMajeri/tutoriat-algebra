\section*{Examen Mincu 2020 Rezolvat}

Fiecare student a primit un \(k\) și un \(l\) diferit.

Vom lua ca exemplu \(k = 73\) și \(l = 38\).

\begin{enumerate}
\item Considerăm corespondențele \(f, g, h \colon \rationals \to \rationals\) date astfel:
\[
f\left(\frac{a}{b}\right) = \frac{a^{2k}}{b^{2l + 1}} \text{ pentru orice } (a, b) \in \integers \times \integers^*,
\]
\[
g\left(\frac{a}{b}\right) = \frac{a^{2k}}{b^{2l + 1}} \text{ pentru orice } (a, b) \in \integers \times \integers^* \text{ pentru care } (a, b) = 1,
\]
\[
h\left(\frac{a}{b}\right) = \frac{a^{2k}}{b^{2l + 1}} \text{ pentru orice } (a, b) \in \integers \times \naturals^* \text{ pentru care } (a, b) = 1.
\]

\begin{enumerate}
    \item Care dintre aceste corespondențe este funcție? Justificați!
    \begin{proof}
    În toate aceste cazuri, numitorul este diferit de zero, deci fracțiile care apar sunt bine definite pe întreg domeniul.

    O proprietate a unei funcții este că, atâta timp cât îi dăm același input, o să aibă același rezultat.

    Dacă ne uităm la \(f\), putem avea aceași fracție scrisă în două moduri, de exemplu \(\frac{1}{2} = \frac{2}{4}\), dar rezultatul diferă:
    \[f\left(\frac{1}{2}\right) = \frac{1}{2^{77}} \neq \frac{1}{2^7} = \frac{2^{146}}{4^{77}} = f\left(\frac{2}{4}\right)\]

    Pentru \(g\), observăm că apare o problemă legată de semne. Să zicem că luăm \(\frac{-1}{-1} = \frac{1}{1}\). Atunci avem:
    \[g\left(\frac{-1}{-1}\right) = \frac{1}{-1} \neq \frac{1}{1} = g\left(\frac{1}{1}\right)\]

    La \(h\), modul în care este definită funcția garantează că nu întâmpinăm vreuna dintre probleme. Fracția este ireductibilă și semnul rezultatului este întotdeauna pozitiv.
    \end{proof}

    \item Pentru acelea dintre ele care sunt funcții, precizați (cu justificare!) dacă sunt sau nu injective, respectiv surjective.

    \begin{proof}
    Pentru \(h\):
    \begin{itemize}
        \item nu este injectivă, dacă luăm \(\frac{-1}{1}\) și \(\frac{1}{1}\) obținem același rezultat, dar parametrii sunt diferiți.
        \item nu este surjectivă, deoarece rezultatul este mereu pozitiv, deci nu acoperă tot \(\rationals\).
    \end{itemize}
    \end{proof}
\end{enumerate}

\item Considerăm grupul \(G = \integersmod{k} \times S_l\) (în cazul nostru \(\integersmod{73} \times S_{38}\)).
\begin{enumerate}
    \item Decideți dacă \(G\) este sau nu ciclic.
    \begin{proof}
    Vom arată că nu este ciclic bazându-ne pe faptul că \(S_n\) pentru \(n \geq 3\) nu este ciclic.

    Să presupunem prin reducere la absurd că \(G\) este ciclic, deci există un generator \((\widehat{a}, \tilde{b}) \in G\). Asta implică că \(n(\widehat{a}, \tilde{b})\) atinge toate valorile lui \(G\), deci și \(n \widehat{a}\) și \(\tilde{b}^n\) ating, eventual, toate valorile din \(\integersmod{73}\), respectiv din \(S_{38}\).

    Deci \(\tilde{b}\) este un generator pentru \(S_{38}\), deci \(S_{38}\) este ciclic. Dar asta contrazice faptul că \(S_{38}\) \href{https://math.stackexchange.com/a/1196364/388180}{nu este ciclic}.
    \end{proof}

    \item Determinați \(\mathop{Hom}(\rationals, G)\).
    \begin{proof}
    Fie \(f \colon \rationals \to G\) un morfism. Notăm \(f(1) = a = (\widehat{x}, \tilde{y}) \in G\).

    Atunci
    \[f(2) = f(1 + 1) = f(1) + f(1) = (\widehat{x}, \tilde{y}) + (\widehat{x}, \tilde{y}) = (\widehat{x} + \widehat{x}, \tilde{y} \circ \tilde{y}) = 2a\]

    Analog
    \[f(-1) = -f(1) = - (\widehat{x}, \tilde{y}) = (-\widehat{x}, \tilde{y}^{-1}) = -a\]

    Deci \(f(n) = n a, \forall n \in \integers\).

    De asemenea, avem
    \[a = f(1) = f\left(\frac{1}{2} + \frac{1}{2}\right) = 2 f\left(\frac{1}{2}\right)\]

    Dacă ne uităm la numerele de forma \(\frac{1}{k}\), obținem asemănător că
    \[a = k f\left(\frac{1}{k}\right), \forall k \in \naturals^*\]

    Atunci, pentru \(k = 73\) avem
    \[a = 73 f\left(\frac{1}{73}\right) \implies \widehat{x} = 73 \cdot \dots = \widehat{0}\]

    Facem același lucru pentru \(k = \abs{S_{38}}\) și obținem că \(\tilde{y} = \tilde{1}\) (permutarea identică).

    Deci \(a = (\widehat{0}, \tilde{1})\), iar \(f(q) = (\widehat{0}, \tilde{1}), \forall q \in \rationals\).
    \end{proof}

    \item Determinați un subgrup \(H\) normal, propriu și netrivial al lui \(G\).
    \begin{proof}
    \(\integersmod{73}\) este un subgrup normal al lui \(\integersmod{73}\).

    \(A_{38}\) (subgrupul permutărilor pare) este un subgrup normal, propriu și netrivial al lui \(S_{38}\).

    \(H = \integersmod{73} \times A_{38}\) este un subgrup normal, propriu și netrivial al lui \(G\).
    \end{proof}

    \item Descrieți, eventual până la izomorfism, grupul factor \(G/H\).
    \begin{proof}
    Când factorizăm \(S_{38}\) prin \(A_{38}\) obținem două clase de resturi: clasa permutărilor pare și clasa permutărilor impare. Având exact două elemente, grupul factor este izomorf cu \(\integersmod{2}\).

    \[
        \frac{G}{H} = \frac{\integersmod{73} \times S_{38}}{\integersmod{73} \times A_{38}} \cong \frac{\integersmod{73}}{\integersmod{73}} \times \frac{S_{38}}{A_{38}} \cong \set{1} \times \integersmod{2} \cong \integersmod{2}
    \]
    \end{proof}
\end{enumerate}

\item
\begin{enumerate}
    \item Determinați numărul elementelor de ordin \(10k\) din grupul \(\integersmod{2020k}\).

    \begin{proof}
    Trebuie să găsim numărul elementelor de ordin \(730\) din grupul \(\integersmod{2020 \cdot 73}\).

    Observăm că \(730 \mid (2020 \cdot 73)\), deci există elemente de acest ordin.

    Ne folosim de raționamentul de \href{http://facstaff.cbu.edu/~wschrein/media/M402\%20Notes/M402L54.pdf}{aici}, care ne spune că numărul cerut este fix \(\phi(730)\), adică numărul de numere de la \(1\) la \(729\) inclusiv prime față de \(730\).

    Putem calcula \(\phi(730)\) foarte rapid descompunând numărul în factori primi: \(730 = 2 \cdot 5 \cdot 73\). Atunci \(\phi(730) = \phi(2 \cdot 5 \cdot 73)\), care \href{https://www.wolframalpha.com/input/?i=phi\%28730\%29}{după multe calcule} iese ca fiind \(288\).
    \end{proof}

    \item Considerăm permutarea \(\sigma\) a literelor alfabetului românesc scrisă ca produs de cicluri astfel: luați (toate) numele și toate prenumele dvs. (așa cum apar în actul de identitate, fără inițiala tatălui, dar cu diacritice) și scrieți-le pe un rând, fără spații. Descompuneți apoi șirul de caractere obținut în blocuri, cu ajutorul parantezelor, închizând fiecare paranteză exact înaintea literei care ar genera o primă repetiție în blocul închis de acea paranteză.

    De exemplu, numele Dulgeru Iancu R.D. Mihaela Florica generază permutarea
    \[\sigma = (dulger)(uianc)(umihael)(afloric)(a)\]

    Descompuneți \(\sigma\) în produs de transpoziții și în produs de cicluri disjuncte. Calculați \(\sigma^3\), \(\sigma^{-1}\), \(\varepsilon(\sigma)\), \(\ord(\sigma)\) și \(\sigma^{2020}\).

    \begin{proof}
    Vom lucra cu permutarea dată ca exemplu.

    Descompunerea ca transpoziții se poate face destul de ușor, doar spargem ciclurile:
    \begin{gather*}
        \sigma = (du)(ul)(lg)(ge)(er) \\
        (ui)(ia)(an)(nc) \\
        (um)(mi)(ih)(ha)(ae)(el) \\
        (af)(fl)(lo)(or)(ri)(ic)
    \end{gather*}
    Nu am mai scris ultimul ciclu \((a)\) pentru că fiind un ciclu cu un singur element, era practic permutarea identică.

    Deși permutarea este deja descompusă în cicluri, acestea \textbf{nu} sunt disjuncte. Trebuie să înmulțim pe rând transpozițiile, urmărim parcursul fiecărei litere să vedem ce cicluri disjuncte obținem.

    De exemplu, dacă am avea \((a b c)(a d)\), am observa că:
    \begin{itemize}
        \item \(a\) se duce în \(d\), care nu e afectat de \((a b c)\)
        \item \(b\) nu e afectat de \((a d)\), se duce în \(c\)
        \item \(c\) nu e afectat de \((a d)\), se duce în \(a\)
        \item \(d\) se duce în \(a\), care se duce în \(b\)
    \end{itemize}
    Deci scris ca ciclu disjunct ar fi \((a d b c)\).

    Dacă la unul dintre acești pași ne întoarcem la începutul ciclului pe care îl calculăm, deschidem un nou ciclu cu una dintre literele rămase și continuăm algoritmul.

    Efectuând calcule asemănătoare, obținem pe rând:
    \begin{gather*}
        (dulger)(uianc) = (duianclger) \\
        (duianclger)(umihael) = (dumar)(ihncl)(ge) \\
        (dumar)(ihncl)(ge)(afloric) = (dumafilo)(hncr)(ge)
    \end{gather*}
    Avem 3 cicluri disjuncte de lungimi 8, 4, și 2.

    Pentru a calcula \(\sigma^3\), e suficient să parcurgem din 3 în 3 ciclurile:
    \[\sigma^3 = (dalufomi) (hrcn) (ge)\]

    Pentru a calcula \(\sigma^{-1}\), citim invers ciclurile:
    \[\sigma^{-1} = (olifamud) (rcnh) (eg)\]

    Pentru a calcula \(\varepsilon(\sigma)\), numărăm câte transpoziții avem: 21, deci permutarea este impară.

    Pentru a calcula \(\ord(\sigma)\), calculăm cel mai mic multiplu comun al lui 8, 4 și 2, adică 8.

    Pentru a calcula \(\sigma^{2020}\), observăm că \(2 \mid 2020\) și \(4 \mid 2020\), deci ciclurile de aceste lungime dispar. De asemenea, \(2020 = 2016 + 4\), deci este suficient să calculăm puterea a 4-a a ciclului de lungime 8:
    \[\sigma^{2020} = (df)(ao)(lm)(ui)\]
    \end{proof}
\end{enumerate}
\end{enumerate}
