\begin{exercise}
Fie \(Q = \Set{ 1, -1, i, -i, j, -j, k, -k }\) grupul cuaternionilor.
\begin{enumerate}
    \item Arătați că \(\Set{ 1, -1 } \leq Q\)
    \item Arătați că \(\Set{ 1, -1 } \trianglelefteq Q\)
    \item Descrieți grupul factor \(\frac{Q}{\Set{ 1, -1 }}\)
\end{enumerate}
\end{exercise}
\begin{proof}
Enunțăm aici regulile de calcul pentru cuaternioni:
\[
\begin{array}{c|c|c|c|c}
    \cdot & 1 & i & j & k \\
    \hline
    1 & 1 & i & j & k \\
    \hline
    i & i & -1 & k & -j \\
    \hline
    j & j & -k & -1 & i \\
    \hline
    k & k & j & -i & -1
\end{array}
\]

Vom nota \(A = \Set{ -1, 1 }\) ca să nu ne tot repetăm.
\begin{enumerate}
    \item Mai întâi arătăm că \(A\) este parte stabilă în raport cu operația „\(\cdot\)”:
    \begin{align*}
        1 \cdot 1 &= 1 \in A \\
        1 \cdot (-1) &= -1 \in A \\
        (-1) \cdot 1 &= -1 \in A \\
        (-1) \cdot (-1) &= 1 \in A
    \end{align*}

    \(A\) este finită deoarece \(\abs{A} = 2\).

    Avem o propoziție care ne zice că \(A\) este subgrup.

    \item Construim clasele de resturi la stânga și la dreapta pentru \(A\) și vedem că acestea se potrivesc unu-la-unu:
    \[
    \begin{aligned}[c]
        1 \cdot A &= -1 \cdot A \\
        i \cdot A &= -i \cdot A \\
        j \cdot A &= -j \cdot A \\
        k \cdot A &= -k \cdot A
    \end{aligned}
    \quad
    \begin{aligned}[c]
        = \Set{ 1, -1 } = \\
        = \Set{ i, -i } = \\
        = \Set{ j, -j } = \\
        = \Set{ k, -k } =
    \end{aligned}
    \quad
    \begin{aligned}[c]
        A \cdot 1 &= A \cdot (-1) \\
        A \cdot i &= A \cdot (-i) \\
        A \cdot j &= A \cdot (-j) \\
        A \cdot k &= A \cdot (-k)
    \end{aligned}
    \]

    Deci \(\Set{ 1, -1 }\) este subgrup normal.

    \item Grupul factor cerut este format din
    \[
        \frac{Q}{\Set{1, -1}} = \Set{ \Set{1, -1}, \Set{i, -i}, \Set{j, -j}, \Set{k, -k} }
    \]

    Vom nota cu \(\widehat{x}\) clasa corespunzătoare elementului \(x \in \Set{ 1, i, j, k }\).
    Obținem o descriere echivalentă a grupului factor:
    \[
        \frac{Q}{\Set{1, -1}} = \Set{ \widehat{1}, \widehat{i}, \widehat{j}, \widehat{k} }
    \]

    Fiind un grup cu 4 elemente, trebuie să fie izomorf fie cu \(\integersmod{4}\) fie cu \(\integersmod{2} \times \integersmod{2}\).
    Ne putem da seama mai ușor cu care este izomorf dacă construim tabelul:
    \[
    \begin{array}{c|c|c|c|c}
         \cdot & \widehat{1} & \widehat{i} & \widehat{j} & \widehat{k} \\
         \hline
         \widehat{1} & \widehat{1} & \widehat{i} & \widehat{j} & \widehat{k} \\
         \hline
         \widehat{i} & \widehat{i} & \widehat{1} & \widehat{k} & \widehat{j} \\
         \hline
         \widehat{j} & \widehat{j} & \widehat{k} & \widehat{1} & \widehat{i} \\
         \hline
         \widehat{k} & \widehat{k} & \widehat{j} & \widehat{i} & \widehat{1}
    \end{array}
    \]

    Se observă ușor că acest tabel corespunde lui \((\integersmod{2} \times \integersmod{2}, +)\).
\end{enumerate}
\end{proof}

\begin{exercise}
Definiția noțiunea de transpoziție și demonstrați că orice transpoziție este o permutare impară.
\end{exercise}
\begin{proof}
O transpoziție este un ciclu de lungime doi.
Ne va ajuta în demonstrație să scriem cum arată la cazul general o transpoziție.

Fie o transpoziție \(\sigma \in S_n\). Aceasta se scrie ca
\[
    \sigma = \begin{pmatrix}
        1 & 2 & \dots & i & \dots & j & \dots & n - 1 & n \\
        1 & 2 & \dots & j & \dots & i & \dots & n - 1 & n
    \end{pmatrix}
\]
unde \(1 \leq i < j \leq n\), și \(\sigma(k) = k\) pentru orice \(k\) diferit de \(i\) și \(j\).

O \emph{inversiune} a unei permutări este orice pereche de indici \(i, j\) cu \(i < j\) dar \(\sigma(i) > \sigma(j)\).
Prin definiție, o permutare se numește \emph{impară} dacă are un număr impar de inversiuni.
Deci trebuie să numărăm inversiunile dintr-o transpoziție pentru a vedea dacă este impară.

Să scriem din nou partea care ne interesează din transpoziție:
\[
    \sigma = \begin{pmatrix}
        \dots & i - 1 & i & i + 1 & \dots & j - 1 & j & j + 1 & \dots \\
        \dots & i - 1 & j & i + 1 & \dots & j - 1 & i & j + 1 & \dots
    \end{pmatrix}
\]

Să numărăm inversiunile cu un capăt în \(i\): avem \(i < i + 1\) dar \(j > i + 1\), \(i < i + 2\) dar \(j > i + 2\) etc. De la \(i\) la \(j - 1\) inclusiv avem un total de \(j - i\) inversiuni.

Analog numărăm inversiunile cu un capăt în \(j\): avem \(i + 1 < j\) dar \(i + 1 > i\), \(i + 2 < j\) dar \(i + 2 > i\) etc. Aici mai avem încă \(j - i\) inversiuni.

Mai trebuie să numărăm și inversiunea \((i, j)\). În total avem un număr impar de inversiuni:
\[
    (j - i) + (j - i) + 1 = 2 (j - i) + 1 = 2m + 1
\]
\end{proof}

\begin{exercise}
Demonstrați că
\[
    S_n = \generatedby{(1, 2), (2, 3), (3, 4), \dots, (n - 1, n)}
\]
(adică că pornind de la acele transpoziții putem genera orice permutare de ordin \(n\))
\end{exercise}
\begin{proof}
De obicei când avem o submulțime și vrem să arătăm că aceasta poate genera tot grupul,
este suficient să arătăm că poate genera un generator/sistem de generatori știut al grupului.

La permutări, știm că mulțimea tuturor transpozițiilor poate genera orice permutare.
Dacă reușim să plecăm de la transpozițiile date și să obținem orice transpoziție, am rezolvat problema.

Să vedem ce se întâmplă când facem următoarea compunere de permutări:
\[
    (1, 2) \circ (2, 3) \circ (1, 2) = (1, 3)
\]
Luăm permutarea generată și calculăm:
\[
    (1, 3) \circ (3, 4) \circ (1, 3) = (1, 4)
\]
Prin inducție, se poate arăta că astfel generăm toate transpozițiile de forma \((1, i)\) cu \(\forall i \in \overline{2, n}\).

Acum încercăm următoarele:
\begin{align*}
    (1, 2) \circ (1, 3) \circ (1, 2) = (2, 3) \\
    (1, 2) \circ (1, 4) \circ (1, 2) = (2, 4) \\
    (1, 2) \circ (1, 5) \circ (1, 2) = (2, 5)
\end{align*}
Generalizând:
\[
    (1, 2) \circ (1, i) \circ (1, 2) = (2, i), \forall i \in \overline{3, n}
\]
Putem obține orice transpoziție:
\[
    (1, i) \circ (1, j) \circ (1, i) = (i, j), \forall i, j \in \overline{3, n}
\]

Din moment ce mulțimea transpozițiilor generează \(S_n\), am demonstrat că
\[
    \Set{(1, 2), (2, 3), \dots, (n - 1, n)}
\]
generează tot \(S_n\).
\end{proof}
