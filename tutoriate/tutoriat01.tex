\begin{exercise}
Scrieți elementele mulțimii \(\powerset(\powerset(\powerset(\varnothing)))\).
\end{exercise}

\begin{proof}
\begin{align*}
    \powerset(\varnothing) &= \Set{ \varnothing } \\
    \powerset(\powerset(\varnothing)) &= \Set{ \varnothing, \set{ \varnothing } } \\
    \powerset(\powerset(\powerset(\varnothing))) &= \Set{ \varnothing, \Set{ \varnothing }, \Set{\Set{\varnothing}}, \Set{ \varnothing, \Set{ \varnothing } } }
\end{align*}
\end{proof}

\begin{exercise}
Arătați că relația de \href{https://en.wikipedia.org/wiki/Modular_arithmetic#Definition_of_congruence_relation}{congruență modulo \(n\)} este relație de echivalență, folosind definiția.
\end{exercise}
\begin{proof}
Fie \(n \in \naturals^*\) fixat. Atunci spunem că \(a \equiv b \mod n\) dacă \(n \mid (a - b)\).

Pentru a demonstra că este relație de echivalență, trebuie să demonstrăm că este \emph{reflexivă}, \emph{simetrică} și \emph{tranzitivă}.

\begin{enumerate}
    \item Fie \(a \in \naturals\). Atunci \(n \mid (a - a) = 0\). Deci \(a \equiv a \mod n\). Deci \(\equiv\) este reflexivă.
    \item Fie \(a, b \in \naturals\) cu \(a \equiv b \mod n\). Din definiție, \(n \mid (a - b)\). Atunci \(n \mid - (a - b)\). De unde rezultă că \(n \mid (b - a)\). Deci \(b \equiv a \mod n\). Deci \(\equiv\) este simetrică.
    \item Fie \(a, b, c \in \naturals\) cu \(a \equiv b \mod n\) și \(b \equiv c \mod n\). Din definiție avem că \(n \mid (a - b)\) și \(n \mid (b - c)\). Atunci facem suma și avem că \(n \mid ((a - b) + (b - c)) \implies n \mid (a - c)\). Deci \(a \equiv c \mod n\). Deci \(\equiv\) este tranzitivă.
\end{enumerate}
Din acestea rezultă că \(\equiv\) este relație de echivalență.
\end{proof}

\begin{exercise}
Demonstrați că relația \(x \rhoequiv y \iff x^2 + 7x = y^2 + 7y\) este de echivalență.
\end{exercise}
\begin{proof}
Demonstrația este similară cu cea de la exercițiul precedent, iar proprietățile decurg din faptul că egalitatea este reflexivă, simetrică și tranzitivă.
\end{proof}

\begin{exercise}
Fie \(A\) și \(A'\) submulțimi ale lui T. Arătați că:
\begin{enumerate}
    \item \(\chi_{A \cap {A'}} = \chi_A \cdot \chi_{A'}\)
    \item \(\chi_{A \cup {A'}} = \chi_A + \chi_{A'} - \chi_A \cdot \chi_{A'}\)

    În particular, dacă \(A\) și \(A'\) sunt disjuncte avem că \(X_{A \cup {A'}} = \chi_A + \chi_{A'}\).
    \item \(\chi_{A \setminus A'} = \chi_A \cdot (1 - \chi_{A'})\)
\end{enumerate}
\end{exercise}
\begin{proof}
Putem demonstra aceste egalități construind tabelul de valori pentru funcțiile \(\chi\).
\begin{enumerate}
    \item
    \[
    \begin{array}{cccc}
        \chi_A & \chi_{A'} & \chi_{A \cap A'} & \chi_A \cdot \chi_{A'} \\
        0 & 0 & 0 & 0 \cdot 0 = 0 \\
        0 & 1 & 0 & 0 \cdot 1 = 0 \\
        1 & 0 & 0 & 1 \cdot 0 = 0 \\
        1 & 1 & 1 & 1 \cdot 1 = 1
    \end{array}
    \]

    \item
    \[
    \begin{array}{cccc}
         \chi_A & \chi_{A'} & \chi_{A \cup A'} & \chi_A + \chi_{A'} - \chi_A \cdot \chi_{A'}  \\
         0 & 0 & 0 & 0 + 0 - 0 \cdot 0 = 0 \\
         0 & 1 & 1 & 0 + 1 - 0 \cdot 1 = 1 \\
         1 & 0 & 1 & 1 + 0 - 1 \cdot 0 = 1 \\
         1 & 1 & 1 & 1 + 1 - 1 \cdot 1 = 1
    \end{array}
    \]

    \item
    \[
    \begin{array}{cccc}
         \chi_A & \chi_{A'} & \chi_{A \setminus A'} & \chi_A \cdot (1 - \chi_{A'})  \\
         0 & 0 & 0 & 0 \cdot (1 - 0) = 0 \\
         0 & 1 & 0 & 0 \cdot (1 - 1) = 0 \\
         1 & 0 & 1 & 1 \cdot (1 - 0) = 1 \\
         1 & 1 & 0 & 1 \cdot (1 - 1) = 0
    \end{array}
    \]
\end{enumerate}
\end{proof}

\begin{exercise}
Dați exemplu de funcții \(\function{f, g}{\naturals}{\naturals}\) cu proprietatea că \(g \circ f = 1_{\naturals}\), dar \(g\) nu este injectivă, iar \(f\) nu este surjectivă.
\end{exercise}
\begin{proof}
O pereche de funcții care îndeplinesc aceste condiții sunt
\[
g(n) = \begin{cases}
0, &\text{ dacă } n = 0 \\
n - 1, &\text{ dacă } n \geq 1
\end{cases}
\]
\[
f(n) = n + 1, \forall n \in \naturals
\]
\end{proof}
