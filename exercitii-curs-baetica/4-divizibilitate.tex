\section{Aritmetica în \texorpdfstring{\(\integers\)}{Z} și \texorpdfstring{\(K[X]\)}{K[X]}}

\subsection{Divizibilitate. Algoritmul lui Euclid}

\begin{exercise}[1.15]
Calculați \((24, 54)\) în \(\integers\) cu algoritmul lui Euclid.
\end{exercise}
\begin{proof}
Împărțim cu rest pe 54 la 24 (dacă am urma exact algoritmul ar trebui să împărțim pe 24 la 54 mai întâi, dar apoi tot la pasul acesta ajungem):
\[
    54 = 24 \cdot 2 + 6
\]
Acum împărțim cu rest pe 24 la 6:
\[
    24 = 6 \cdot 4 + 0
\]
Deoarece restul obținut este 0, algoritmul se termină. În concluzie, un c.m.m.d.c. al lui 24 și 54 este 6.
\end{proof}

\begin{exercise}[1.16]
Calculați \((X^4 - 4X^3 + 1, X^3 - 3X^2 + 1)\) în \(\reals[X]\) cu algoritmul lui Euclid.
\end{exercise}
\begin{proof}
Notăm \(f = X^4 - 4X^3 + 1\), \(g = X^3 - 3X^2 + 1\).
Împărțim cu rest pe \(f\) la \(g\):
\[
\polylongdiv[style=B]{X^4 - 4X^3 + 1}{X^3 - 3X^2 + 1}
\]
Deci \(f = (X - 1) g + (- 3X^2 - X + 2)\). Notăm \(r_1 = -3X^2 - X + 2\). Acum împărțim cu rest pe \(g\) la \(r_1\):
\[
\polylongdiv[style=B]{X^3 - 3X^2 + 1}{-3X^2 - X + 2}
\]
Deci \(g = \left(-\frac{1}{3} X + \frac{10}{9}\right) r_1 + \left(\frac{16}{9} X - \frac{11}{9}\right)\).

Dacă înmulțim un polinom cu un element inversabil din inelul de coeficienți, noul polinom este asociat în divizibilitate cu cel precedent. Deci o să înmulțim cu numitorul comun (care este inversabil, deoarece lucrăm în \(\reals\)) și notăm \(r_2 = 16 X - 11\) pentru a simplifica calculele.

Împărțim cu rest pe \(r_1\) la \(r_2\):
\[
\polylongdiv[style=B]{-3X^2 - X + 2}{16 X - 11}
\]
Deci \(r_1 = \left(-\frac{3}{16} X - \frac{49}{256}\right) r_2 + (-\frac{27}{256})\).
O să înmulțim restul obținut cu \(-\frac{256}{27}\). Notăm \(r_3 = 1\).

Împărțim cu rest pe \(r_2\) la \(r_3\):
\[
\polylongdiv[style=B]{16 X - 11}{1}
\]
De unde rezultă \(r_2 = 1 \cdot r_3 + 0\). Ultimul rest nenul este \(1\), deci cele două polinoame sunt prime între ele: \((X^4 - 4X^3 + 1, X^3 - 3X^2 + 1) = 1\).
\end{proof}

\begin{exercise}[1.22]
Determinați \((X^2 - 1) \rationals[X] \cap (X^3 - 1) \rationals[X]\) și \((X^2 - 1) \rationals[X] + (X^3 - 1) \rationals[X]\).
\end{exercise}
\begin{proof}
Ne folosim de proprietățile din curs: intersecția a două ideale generate de un singur element este idealul generat de c.m.m.m.c.-ul al celor două elemente, respectiv pentru adunare de c.m.m.d.c.-ul celor două elemente.

Pentru a fi mai ușor să găsim c.m.m.d.c.-ul și c.m.m.m.c.-ul, descompunem polinoamele în factori ireductibili din \(\rationals[X]\):
\begin{align*}
    X^2 - 1 &= (X - 1)(X + 1) \\
    X^3 - 1 &= (X - 1)(X^2 + X + 1)
\end{align*}
Obținem \([X^2 - 1, X^3 - 1] = (X - 1)(X + 1)(X^2 + X + 1)\) și \((X^2 - 1, X^3 - 1) = X - 1\).

Deci
\begin{align*}
    (X^2 - 1) \rationals[X] \cap (X^3 - 1) \rationals[X] &= ((X - 1)(X + 1)(X^2 + X + 1)) \rationals[X] \\
    (X^2 - 1) \rationals[X] + (X^3 - 1) \rationals[X] &= (X - 1) \rationals[X]
\end{align*}
\end{proof}

\subsection{Elemente prime. Elemente ireductibile}

\begin{exercise}[2.10]
Determinați polinoamele ireductibile de grad \(\leq 5\) din \(\integersmod{2}[X]\).
\end{exercise}
\begin{proof}
Sunt cel mult \(2^6 = 64\) polinoame de grad \(\leq 5\) în \(\integersmod{2}[X]\). Ca să nu fie nevoie să le încercăm pe toate, facem câteva observații ajutătoare:
\begin{enumerate}
    \item Polinoamele de grad \(\leq 1\) sunt sigur ireductibile: \(\widehat{0}, \widehat{1}, \widehat{X}, \widehat{X + 1}\).
    
    \item Polinoamele de grad \(\geq 2\) care nu au un termen liber \(\widehat{1}\) sunt sigur reductibile, pentru că se divid prin \(\widehat{X}\).
    
    \item Dacă un polinom are rădăcina \(\alpha\), atunci sigur este reductibil (din Bézout, se divide prin \(X - \alpha\)). Deoarece lucrăm în \(\integersmod{2}\), este suficient să încercăm să înlocuim \(X\) cu \(\widehat{0}\) și \(\widehat{1}\).
    
    \item Pentru că \(\integersmod{2}\) este corp de caracteristică 2 (deoarece \(\widehat{1} + \widehat{1} = \widehat{0}\)), avem că \((a + b)^2 = a^2 + b^2\) (sau mai general, \((\sum x)^2 = \sum (x^2)\)).
    
    Deci orice polinom din \(\integersmod{2}[X]\) cu toate monoamele de grad par este reductibil. De exemplu, \(X^4 + X^2 + \widehat{1} = (X^2 + X + \widehat{1})^2\).
    
    \item În \(\integersmod{2}[X]\), dacă suma coeficienților unui polinom este un număr par, atunci acel polinom sigur are rădăcina \(\widehat{1}\), deci este reductibil.
    
    \item Pe măsură ce construim lista, dacă observăm că un polinom este egal cu produsul altor polinoame deja scrise, atunci este reductibil.
\end{enumerate}

Lista polinoamelor ireductibile de grad \(\leq 5\) din \(\integersmod{2}[X]\):
\begin{itemize}
    \item Grad \(\leq 1\): \(0\), \(1\), \(X\), \(X + 1\)
    \item Grad 2: \(X^2 + X + 1\)
    \item Grad 3: \(X^3 + X^2 + 1\), \(X^3 + X + 1\)
    \item Grad 4: \(X^4 + X^3 + X^2 + X + 1\), \(X^4 + X^3 + 1\), \(X^4 + X + 1\)
    \item Grad 5: \(
        X^5 + X^2 + 1,
        X^5 + X^3 + 1,
        X^5 + X^3 + X^2 + X + 1,
        X^5 + X^4 + X^2 + X + 1,
        X^5 + X^4 + X^3 + X + 1,
        X^5 + X^4 + X^3 + X^2 + 1
    \)
\end{itemize}
\end{proof}

\begin{exercise}[2.11]
Descompuneți polinomul \(f = X^{56} - X^{49} - X^7 + \widehat{1}\) în produs de polinoame ireductibile în \(\integersmod{7}[X]\).
\end{exercise}
\begin{proof}
Notăm \(Y = X^7\). Polinomul inițial este \(f = Y^8 - Y^7 - Y + \widehat{1}\).

Observăm că \(\widehat{1}\) este o rădăcină (multiplă) a polinomului. Tot împărțim prin \(Y - \widehat{1}\) și obținem:
\begin{align*}
    f &= Y^8 - Y^7 - Y + 1 \\
    &= (Y - \widehat{1}) (Y^7 - \widehat{1}) \\
    &= (Y - \widehat{1})^2 (Y^6 + Y^5 + Y^4 + Y^3 + Y^2 + Y + \widehat{1}) \\
    &= (Y - \widehat{1})^3 (Y^5 + \widehat{2} Y^4 + \widehat{3} Y^3 + \widehat{4} Y^2 + \widehat{5} Y - \widehat{1}) \\
    &= (Y - \widehat{1})^4 (Y^4 + \widehat{3} Y^3 + \widehat{6} Y^2 + \widehat{3} Y + \widehat{1}) \\
    &= (Y - \widehat{1})^5 (Y^3 + \widehat{4} Y^2 + \widehat{3} Y - \widehat{1}) \\
    &= (Y - \widehat{1})^6 (Y^2 + \widehat{5} Y + \widehat{1}) \\
    &= (Y - \widehat{1})^7 (Y - \widehat{1}) \\
    &= (Y - \widehat{1})^8
\end{align*}

Deci polinomul inițial este \(f = (Y - \widehat{1})^8 = (X^7 - \widehat{1})^8\). Pe baza calculelor deja efectuate pentru \(Y\), acesta este egal cu \(((X - \widehat{1})^7)^8 = (X - \widehat{1})^{56}\).
\end{proof}

\begin{exercise}[2.17]
Determinați c.m.m.d.c. și c.m.m.m.c. pentru polinoamele \(f = (X - 1)(X^2 - 1)(X^3 - 1)(X^4 - 1)\) și \(g = (X + 1)(X^2 + 1)(X^3 + 1)(X^4 + 1)\) din \(\rationals[X]\).
\end{exercise}
\begin{proof}
Deoarece polinoamele sunt deja scrise ca produs de alte polinoame, începem prin a descompune cele două polinoame în produs de factori ireductibili peste \(\rationals[X]\). Avem că
\begin{align*}
    f &= (X - 1)(X - 1)(X + 1)(X - 1)(X^2 + X + 1)(X - 1)(X + 1)(X^2 + 1) \\
    &= (X - 1)^3 (X + 1)^2 (X^2 + 1) (X^2 + X + 1) \\
    g &= (X + 1)(X^2 + 1)(X^3 + 1)(X^4 + 1)
\end{align*}

Pentru a găsi c.m.m.d.c.-ul, luăm toți factorii primi comuni la puterea cea mai mică.
\[
(f, g) = (X + 1)(X^2 + 1)
\]

Pentru c.m.m.m.c., luăm toți factorii primi o singură dată, la puterea cea mai mare.
\[
[f, g] = (X - 1)^3 (X + 1)^2 (X^2 + 1)(X^3 + 1)(X^2 + X + 1)(X^4 + 1)
\]
\end{proof}

\subsection{Teorema fundamentală a algebrei}

\begin{exercise}[3.7]
Arătați că polinomul \(X^n - 2\) este ireductibil în \(\rationals[X]\) pentru orice \(n \geq 1\).
\end{exercise}
\begin{proof}
~
\begin{itemize}
    \item Pentru \(n = 1\), \(X - 2\) este polinom de grad 1, deci ireductibil.
    
    \item Pentru \(n = 2\), rădăcinile polinomului sunt \(\pm \sqrt{2}\), care nu aparțin lui \(\rationals\), deci polinomul este ireductibil.
    
    \item Pentru \(n \geq 3\), facem observația că polinomul este reductibil peste \(\complex[X]\), soluțiile fiind rădăcinile de ordin \(n\) ale lui 2.
    Deci \(X^n - 2\) mai poate fi scris ca
    \[
        X^n - 2 = (X - \varepsilon_0) \dots (X - \varepsilon_{n - 1})
    \]
    unde \(\varepsilon_k = \sqrt[n]{2} (\cos \frac{2 k \pi}{n} + i \sin \frac{2 k \pi}{n})\). De aici rezultă că \((-1)^n \cdot \varepsilon_0 \cdot \dots \cdot \varepsilon_n = -2\).
    
    Să presupunem că putem scrie \(X^n - 2\) ca produs de polinoame ireductibile peste \(\rationals[X]\):
    \[
        X^n - 2 = (X^{k_1} + \dots + a_{1}) \dots (X^{k_r} + \dots + a_{r})
    \]
    Toate aceste polinoame trebuie să fie de grad cel puțin 2, pentru că rădăcinile polinomului nu sunt numere raționale. 
    
    Dacă egalăm cele două descompuneri, și ne uităm la termenii liberi care apar când calculăm produse dintre unii termeni, am obține că un produs de numere raționale \(a_{i_1} \dots a_{i_p}\) este egal cu un produs de numere complexe \(\varepsilon_{j_1} \dots \varepsilon_{j_r}\). Deci polinomul nu poate fi descompus peste \(\rationals\).
\end{itemize}
\end{proof}

\begin{exercise}[3.8]
Descompuneți polinomul \(X^{n} - 1, 1 \leq n \leq 6\), în produs de polinoame ireductibile în \(\rationals[X]\), \(\reals[X]\), respectiv \(\complex[X]\).
\end{exercise}
\begin{proof}
~
\begin{itemize}
    \item Pentru \(n = 1\), polinomul \(X - 1\) este deja ireductibil, fiind de grad 1.
    
    \item Pentru \(n = 2\), polinomul \(X^2 - 1\) se descompune ca \((X - 1)(X + 1)\).
    
    \item Pentru \(n = 3\), polinomul \(X^3 - 1\) se descompune ca \((X - 1)(X^2 + X + 1)\). Peste \(\rationals\) și \(\reals\), factorul \(X^2 + X + 1\) este ireductibil, fiind de gradul 2, cu \(\Delta < 0\).
    
    Peste \(\complex\), descompunerea completă este \((X - 1)(X - e^{i \frac{2 \pi}{3}})(X - e^{i \frac{2 \pi}{3}})\).
    
    \item Pentru \(n = 4\), polinomul \(X^4 - 1\) se descompune ca \((X^2 - 1)(X^2 + 1) = (X - 1)(X + 1)(X^2 + 1)\). Aceasta este descompunerea în factori ireductibili peste \(\rationals\) și \(\reals\).
    
    Peste \(\complex\) polinomul se descompune ca \((X - 1)(X + 1)(X - i)(X + i)\).
    
    \item Pentru \(n = 5\), polinomul \(X^5 - 1\) se descompune ca \((X - 1)(X^4 + X^3 + X^2 + X + 1)\). Această descompunere este ireductibilă peste \(\rationals\).
    
    Peste \(\reals\) mai putem descompune polinomul în \((X - 1)(X^2 + \frac{1 - \sqrt{5}}{2} X + 1)(X^2 + \frac{1 + \sqrt{5}}{2} X + 1)\).
    
    Peste \(\complex\) polinomul se descompune complet în \((X - \varepsilon_0) \dots (X - \varepsilon_4)\), unde \(\varepsilon_k = \cos \frac{2 k \pi}{5} + i \sin \frac{2 k \pi}{5}\).
    
    \item Pentru \(n = 6\), putem scrie \(X^6 - 1\) ca \((X^3 - 1)(X^3 + 1)\). Deja am analizat descompunerea lui \(X^3 - 1\).
    
    În ceea ce privește \(X^3 + 1\), acest polinom se descompune în \((X + 1)(X^2 - X + 1)\). Aceasta este descompunerea finală peste \(\rationals\) și \(\reals\), dar peste \(\complex\) mai putem descompune \(X^2 - X + 1\) în \((X - \varepsilon_1)(X - \varepsilon_2)\), unde \(\varepsilon_k = - (\cos \frac{2 k \pi}{3} + i \sin \frac{2 k \pi}{3})\).
\end{itemize}
\end{proof}

\subsection{Teorema chineză a resturilor}

\begin{exercise}[4.1]
Să se afle cea mai mică soluție pozitivă a sistemului de congruențe
\[
\begin{cases}
    x \equiv 5 \mod{18} \\
    x \equiv 27 \mod{35}
\end{cases}
\]
\end{exercise}
\begin{proof}
Deoarece 18 și 35 sunt prime între ele, putem aplica teorema chineză a resturilor.
Începem prin a găsi inversul modular al lui 18 în \(\integersmod{35}\), respectiv al lui 35 în \(\integersmod{18}\) (putem face asta folosind algoritmul lui Euclid extins):
\[
\begin{cases}
    18 \cdot 2 &\equiv 1 \mod{35} \\
    35 \cdot 17 &\equiv 1 \mod{18}
\end{cases}
\]
Atunci \(x = 27 \cdot (18 \cdot 2) + 5 \cdot (35 \cdot 17)\) este o soluție pozitivă a sistemului.
Pentru a găsi cea mai mică soluție pozitivă, calculăm restul împărțirii lui \(x\) la \(18 \cdot 35\):
\[
    27 \cdot (18 \cdot 2) + 5 \cdot (35 \cdot 17) \equiv 167 \mod{18 \cdot 35}
\]
Verificare:
\[
\begin{cases}
    167 &\equiv 5 \mod{18} \\
    167 &\equiv 27 \mod{35}
\end{cases}
\]
\end{proof}

\begin{exercise}[4.2]
Rezolvați sistemul de congruențe
\[
\begin{cases}
6x \equiv 2 \mod{8} \\
5x \equiv 5 \mod{6}
\end{cases}
\]
\end{exercise}
\begin{proof}
Deoarece 8 și 6 nu sunt prime între ele, nu putem aplica teorema chineză a resturilor. Încercăm să mai simplificăm sistemul pentru a găsi o soluție manual.

În prima ecuație putem simplifica cu 2:
\[
\begin{cases}
3x \equiv 1 \mod{4} \\
5x \equiv 5 \mod{6}
\end{cases}
\]

Putem înmulți prima ecuație cu inversul modulo 4 al lui 3, și a doua ecuație cu inversul modulo 6 al lui 5. Sistemul devine:
\[
\begin{cases}
x \equiv 3 \mod{4} \\
x \equiv 1 \mod{6}
\end{cases}
\]

Trebuie să găsim soluția care este de forma \(x = 4p + 3 = 6q + 1\) pentru un \(p, q \in \integers\). De asemenea, este suficient să căutăm soluția printre numerele cuprinse între 0 și \(4 \cdot 6 = 24\).

Observăm că 19 este o soluție:
\[
\begin{cases}
19 \equiv 3 \mod{4} \\
19 \equiv 1 \mod{6}
\end{cases}
\]
De asemenea, toate numerele de forma \(24k + 19, \forall k \in \integers\) sunt soluții.
\end{proof}
